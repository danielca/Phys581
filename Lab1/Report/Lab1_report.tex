\documentclass[twocolumn]{myarticle}

\usepackage{mymacros}
\usepackage{parskip}
\usepackage{hyperref}
\usepackage{listings}

\lstset{%
basicstyle=\small\ttfamily,
columns=flexible,
breaklines=true,
numbers=left,
stepnumber=1,}

\begin{document}

\title{Monte Carlo methods in computational physics}
\author{Casey Daniel and Chris Deimert}
\date{\today}

\maketitle

\section{Introduction}
\label{sec:introduction}

In this report, we explore a number of Monte Carlo numerical methods.
Monte Carlo methods use pseudorandom numbers to explore complicated systems.
These methods tend to converge slowly for simple problems, but can be very efficient for complex problems: especially those with a large number of variables.

\section{Pseudo random numbers}
\label{sec:pseudo_random_numbers}

Monte Carlo methods typically rely heavily on our ability to generate a large quantity of random numbers.
We cannot, of coure, generate truly random numbers on a deterministic computer, so we have to approximate them with pseudo-random numbers (PRN's).
Having a reliable PRN generator is thus a key prerequisite to any Monte Carlo method.
In this section, we will study the linear congruential method (LCM) for generating PRN's.
The LCM generates a sequence of pseudo-random integers with the $ n $th integer given by
\begin{align}
    I_{n} &= \left( A I_n + C \right) \! \! \! \! \! \mod M
\end{align}
($ I_0 $ is called the "seed".)
We can then generate a sequence $ x_n = I_n/M $ of pseudo-random real numbers with $ 0 < x_n < 1 $.

A Fortran 90 module was created to implement the LCM generator and can be seen in Section~\ref{subsec:pseudo_random_numbers_module}.
This code was tested in a number of ways, the code for which is in Section~\ref{subsec:pseudo_random_numbers_main_code} and Section~\ref{subsec:pseudo_random_numbers_plotting_code}.)

The first, simplest test used $ I_0 = 3 $, $ A = 7 $, $ C = 0 $, and $ M = 10 $.
The result is a repeating sequence of numbers:
\begin{align}
    x &= 0.1, 0.7, 0.9, 0.3, 0.1, 0.7, 0.9, 0.3, \ldots
\end{align}
This sequence repeats after only 4 numbers, demonstrating why small values of $ M $ are a poor choice.
The sequence will repeat after at most $ M $ numbers, so we must select a high value of $ M $ in order to obtain a long sequence (though high values of $ M $ do not \emph{guarantee} a long sequence).



\onecolumn

\section{Code}
\label{sec:code}

\subsection{Pseudo random numbers module}
\label{subsec:pseudo_random_numbers_module}

\lstinputlisting[breaklines]{../../Modules/Random_numbers_module.f90}
\vspace{10pt}

\subsection{Pseudo random numbers main code}
\label{subsec:pseudo_random_numbers_main_code}

\lstinputlisting[breaklines]{../Pseudo_RNGs.f90}
\vspace{10pt}

\subsection{Pseudo random numbers plotting code}
\label{subsec:pseudo_random_numbers_plotting_code}

\lstinputlisting[breaklines]{../Pseudo_RNGs.gp}
\vspace{10pt}

\end{document}
