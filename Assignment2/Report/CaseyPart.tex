%4.2.1
We can Also Apply our knowledge of the auto correlation to examine the noise in a signal. We start with a signal of 
\begin{equation}
s(t) = \frac{1}{1-0.9sin(t)}
\end{equation}
as shown in figure \ref{noiseless}. We can then start to add random noise of the form $n(i) = s(i) + \alpha (r_{i} -1)$. where $\alpha$ is a scalable parameter and $r_{i}$ is a random number between 0 and 1. Figures \ref{fig:a01} to \ref{a10} show the effect of the parameter $\alpha$. With 0.1, we see a small amount of noise, however the original signal is still clearly visible. $\alpha = 1$ starts to obscure the original signal a bit, however the signal is still clear. By the time we reach $\alpha = 10$ the noise becomes obscured. There are hints, however it is difficult to see, and the amplitude of the signal is not clear. We can also compare the DFT of the auto correlation functions. We can compare the auto correlation of these signal. Figure \ref{fig:noiseless} shows the Auto Correlation DFT has a very high DC offset, and a very wide curve out to the base noise. By the time we approach figure \ref{fig:a10} we see a very sharp peak centred at the origin. This allows us to look at the DFT of the auto correlation to check for random noise added to the signal. Once the type of noise is known, other filtering methods can be applied to obtain the original singal.

\begin{figure}[ht]
    \centering
    \includegraphics[width=0.9\linewidth]{../Plots/signal_noiseless.pdf}
    \caption{Time Series, DFT, and the Auto Correlation function for the noiseless signal.}
    \label{fig:noiseless}
\end{figure}

\begin{figure}[ht]
    \centering
    \includegraphics[width=0.9\linewidth]{../Plots/signal_a01.pdf}
    \caption{Time Series, DFT, and the Auto Correlation function for the a signal with randomly generated noise scaled to $\alpha = 0.1$.}
    \label{fig:a01}
\end{figure}
\begin{figure}[ht]
    \centering
    \includegraphics[width=0.9\linewidth]{../Plots/signal_a1.pdf}
    \caption{Time Series, DFT, and the Auto Correlation function for the a signal with randomly generated noise scaled to $\alpha = 1$.}
    \label{fig:a1}
\end{figure}
\begin{figure}[ht]
    \centering
    \includegraphics[width=0.9\linewidth]{../Plots/signal_a10.pdf}
    \caption{Time Series, DFT, and the Auto Correlation function for the a signal with randomly generated noise scaled to $\alpha = 10$.}
    \label{fig:a10}
\end{figure}

\section{ODEs}
\subsection{Fourier Transform}
We can use the table provided in to preform the transform on the given equations.
\begin{eqnarray*}
m\ddot{x(t)} + D\dot{x(t)} + kx(t) = n(t) \rightarrow \hat{x}(\omega)(m\omega^{2} + iD\omega + k) = \hat{n}(t) \\
i\hbar \frac{\partial \Psi }{\partial t} + \frac{\hbar^{2}}{2m} \frac{\partial^2 \Psi}{\partial \Psi^2} = 0\rightarrow i\hbar \frac{\partial \hat{\Psi}}{\partial t} - \frac{-\hbar \omega^2}{2m}\hat{\Psi} = 0 \\
\frac{\partial^2 T}{\partial x^2} + \frac{\partial^2 T}{\partial z^2} = \delta (x) \delta(z+a) \rightarrow -\omega^2 \hat{T} + \frac{\partial^2 \hat{T}}{\partial z^2} = e^{-2\pi i \omega}\delta (z+a)
\end{eqnarray*}


%Section 6
\section{General Applications}
\subsection{Heart Beats}

We were abel to examine heart beats. This file contained the number of counts sampled at 125Hz, and a total of 4096 samples. Because our number of samples is a power of 2, we don't need to pad this sample set to be able to preform an FFT. Figure \ref{fig:heart_beat} shows the amplitude spectrum of the heart beats. The two most dominant frequencies, excluding the DC component, are 2.2Hz or about 0.45 beats every second. This corresponds to about to a heart rate of 132 beats per minute. The second largest peak is 3.8Hz, giving 0.26 beats every second. This gives a heart rate of 228 beats per minute. These periods are a bit high for that of a human, but could be for another animal
\begin{figure}[ht]
    \centering
    \includegraphics[width=0.9\linewidth]{../Plots/beats.pdf}
    \caption{The DFT of the heart beat signal.}
    \label{fig:heart_beat}
\end{figure}

\subsection{Solar Flares}
We can also examine sun spots, using data provided by NASA. Figure \ref{fig:solar_spots} shows the number of observed spots per month over several years. Just looking visually see approximately 4 cycles in 500 months, or a period of about 10.5. We have a sample set of 3143 months, and to be able to take advantage of a FFT, we will need to pad up to 4096 samples, the nearest power of 2, with 0's. In the bottom portion of figure \ref{fig:solar_spots} we can see the FFT. There is a strong DC component, as well as a smaller side peak. This peak corresponds with a frequency of 7.56E-3, giving a period of 132.2 months or 11.1 years. Given that this is calculated rather than a visual estimation, gives a more accurate figure.

\begin{figure}[ht]
    \centering
    \includegraphics[width=0.9\linewidth]{../Plots/solarSpots.pdf}
    \caption{Time series and DFT of solar spot cycles}
    \label{fig:solar_spots}
\end{figure}

\subsection{Stock Markets}
\subsection{Dow Jones}
We can also use Fourier Analysis on the stock markets. Using the given data for the Dow Jones Index we can use both the Auto Correlation, and DFT to examine it, as seen in figure \ref{fig:dow}. The DFT shows very does not show any strong peaks, however there is information in the auto correlation. For randomized numbers we expect the auto correlation function to hover around 0 for lag values greater than 1. We see that in this case that there may be some periodic nature to this data.

\begin{figure}[ht]
    \centering
    \includegraphics[width=0.9\linewidth]{../Plots/dow_fft.pdf}
    \caption{Fourier Analysis of the Dow Jones data.}
    \label{fig:dow}
\end{figure}

\subsection{Dow Jones Industrial Average}
We can also look at the dow jones industrial average over the past year. Data was found from \url{http://research.stlouisfed.org/fred2/series/DJIA/downloaddata}. We can then compute the auto correlation for different lags on daily, weekly, and monthly intervals. Using the fortran code seen in the appendix. We can also look at the returns for these same periods, both are summarized in the table \ref{table:DJA}.
\begin{tabular}
\begin{table}{|c|c|c|c|c|}
    \hline
    K && 1 && 2 && 5 && 20 \\ \hline
    Daily Price && 0.975 && 0.955 && 0.877 && 0.608 \\ \hline
    Daily Returns && -0.089 && 0.104 && 0.005 && -0.061 \\ \hline
    Weekly Price && 0.880 && 0.794 && 0.584 && -0.169 \\ \hline
    Weekly returns && -0.174 && 0.066 && 0.048 && 0.023 \\ \hline
    Monthly Price && 0.610 && 0.625 && -0.438 && N/A \\ \hline
    Monthly Returns && -0.540 && -0.287 && -0.566 && N/A \\ \hline 
\end{table}
\caption{Summary of the Auto Correlation of the Dow Jones Industrial Average and the corresponding returns sampled daily, weekly, and monthly over a period of one year}
\label{table:DJA}
\end{tabular}

We can also see the returns as a function of time, shown in figure \ref{fig:returns}. When we look at the stock price, wee see the auto correlation values hover between 0.5 and 1, for all sampling periods. This would indicate that there is something periodic in the stock prices. Once we look at the the returns the stock price gives, we see the auto correlation function hover around 0 for the same sampling periods. This conflicting information can be explained. The stock prices don't have any wild fluctuations. Making the prices easily describable by periodic functions, however the returns have a higher variance, meaning that actual profit that can be gained by these stock prices do kj:g and are more of a random process. 

\begin{figure}[ht]
    \centering
    \includegraphics[width=0.9\linewidth]{../Plots/returns.pdf}
    \caption{Returns of the Dow Jones Industrial Average as a function of time, sampled montly, weekly, and daily.}
    \label{fig:returns}
\end{figure}


\subsection{Global Warming}
One last place we can apply our knowledge of Fourier Analysis is to global warming, using the carbon data provided. Figure \ref{fig:co2_time} shows the time series in the top panel in months. There is a linear trend to this data. We can use gnuplot to fit a linear line to the data, and then remove the trend for analysis. This gives us a trend of $f(t) = 0.118083t + 308.859$, this trend is shown along side the data, while the bottom panel shows the data with the trend removed. From the power spectrum in figure \ref{fig:cos_power} we see a very strong harmonics. This indicates the periodic nature of the carbon emissions. Looking at the bottom panel we see that the as time goes on, there is an additional small trend, this would be an indication of acceleration. From figure \ref{fig:cos_time} we can see a small acceleration in the carbon dioxide levels

According to the source provided Carbon Dioxide starts to become toxic at around 10000ppm. We can use the the linear trend to get a rough estimate of when we will reach these levels. This gives us an estimate of roughly 60000 years before we reach these levels. 


\begin{figure}[ht]
    \centering
    \includegraphics[width=0.9\linewidth]{../Plots/co2_time.pdf}
    \caption{Time series of $CO_{2}$ with the linear trend, and with the linear trend removed}
    \label{fig:co2_time}
\end{figure}

\begin{figure}[ht]
    \centering
    \includegraphics[width=0.9\linewidth]{../Plots/co2_power.pdf}
    \caption{DFT of the $CO_{2}$ Cycles with linear trend removed}
    \label{fig:co2_power}
\end{figure}




