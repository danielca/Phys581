\documentclass[twocolumn]{myarticle}

\usepackage{mymacros}
\usepackage{parskip}
\usepackage{hyperref}
\usepackage{listings}
\usepackage{booktabs}
\usepackage{mathtools}

\lstset{%
basicstyle=\small\ttfamily,
columns=flexible,
breaklines=true,
numbers=left,
stepnumber=1,}

\newcommand{\mat}[1]{\begin{bmatrix}#1\end{bmatrix}}
\newcommand{\sinc}{\text{sinc}}
\renewcommand{\d}{\mathrm{d}}

\begin{document}

\title{Physics 581, Lab 2:\\Hooray for Fourier!}
\author{Casey Daniel and Chris Deimert}
\date{\today}

\maketitle

\section{Introduction}
\label{sec:introduction}

\section{Warm up}
\label{sec:warm_up}

In this section, we take the discrete fourier transform (DFT) of four sampled signals using the fast Fourier transform (FFT) routine from Numerical Recipes.
The resulting power spectra (in dB) are shown in Figure~\ref{fig:warm_up}.

\begin{figure*}[htpb]
    \centering
    \includegraphics[width=0.8\linewidth]{../Plots/Warm_up.pdf}
    \caption{%
        The power spectra of the DFT's of four signals.
        Each signal was sampled at $ t = i/1024 $ for $ i = 1, 2, \ldots, 1024 $.
        The frequencies resulting from the discrete Fourier transform are then $ f = -511, -510, \ldots, 512 $.
    }
    \label{fig:warm_up}
\end{figure*}

In the top left of Figure~\ref{fig:warm_up}, we see the power spectrum of 
\[ 
    \sin(2 \pi \cdot 100 t) + 0.5 \sin (2 \pi \cdot 200 t) . 
\]
The power of the $ \SI{200}{Hz} $ signal is less than the power of the $ \SI{100}{Hz} $ signal, as it should be (the peak is about $ \SI{5}{dB} $ lower).
Also, we see that the spectrum is very sharply peaked, with around a $ \SI{150}{dB} $ power drop over a very short frequency range.

In the top right of Figure~\ref{fig:warm_up}, we use the same signal, but perturb one of the frequencies by $ \SI{0.5}{Hz} $ to get
\[
    \sin(2 \pi \cdot 100.5 t) + 0.5 \sin (2 \pi \cdot 200 t) . 
\]
Here we see a spreading of the power of the $ \SI{100.5}{Hz} $ signal.
This is because the frequencies of the DFT are integer multiples of $ \SI{1}{Hz} $.
In the last case, both sinusoids lined up perfectly with a DFT frequency, leading to very clean peaks. 
In this case, the $ \SI{100.5}{Hz} $ signal lies right in between two frequencies, leading to a spread in the spectrum when it should be very sharply peaked like a delta function.
So the DFT is not as good at capturing frequency components in between the native DFT frequencies.

In the bottom left of Figure~\ref{fig:warm_up}, we see an amplitude modulated signal
\[
    \left[ 2 + \sin(2\pi \cdot 8 t) \right] \cdot \sin ( 2 \pi \cdot 100 t)
\]
Here $ \SI{100}{Hz} $ is the carrier frequency and the message is $ 2 + \sin(2 \pi \cdot 8 t) $.
The spectrum of the message contains three peaks at $ \SI{-2}{Hz} $, $ \SI{0}{Hz} $, and $ \SI{2}{Hz} $.
By multiplying the message with a $ \SI{100}{Hz} $ carrier wave, we shift the message spectrum.
Instead of three peaks centred on $ \SI{0}{Hz} $, it is now three peaks centred on $ \SI{+-100}{Hz} $, with an additional boost to the $ \SI{+-100}{Hz} $ peak because of the carrier wave itself.
This can be seen in the figure, though it is difficult, because the bandwidth of the message is small compared to the carrier frequency.

In the bottom right of Figure~\ref{fig:warm_up}, we see a frequency modulated signal:
\[
    \sin \Big( 2 \pi \cdot 100 \big( 1 + 0.1 \sin (2 \pi \cdot 8 t) \big) t \Big)
\]
The carrier frequency is $ \SI{100}{Hz} $ and the message is proportional to $ t \cdot \sin(2 \pi \cdot 8 t) $.
This is a very wide-band message with a large amplitude, so we would expect the resulting frequency-modulated signal to have an expremely large bandwidth compared with the carrier.
This is why we do not see any significan peaks near $ \SI{100}{Hz} $ or any drop-off at higher frequencies.
Thanks to aliasing and the wide-band nature of the signal, we see a near-uniform mess of power across the frequency range of the DFT.


\section{Code}
\label{sec:code}

%\lstinputlisting[breaklines]{../Traveling.gp}
%\vspace{10pt}

\end{document}
